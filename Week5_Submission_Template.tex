% Options for packages loaded elsewhere
\PassOptionsToPackage{unicode}{hyperref}
\PassOptionsToPackage{hyphens}{url}
\PassOptionsToPackage{dvipsnames,svgnames,x11names}{xcolor}
%
\documentclass[
]{article}

\usepackage{amsmath,amssymb}
\usepackage{iftex}
\ifPDFTeX
  \usepackage[T1]{fontenc}
  \usepackage[utf8]{inputenc}
  \usepackage{textcomp} % provide euro and other symbols
\else % if luatex or xetex
  \usepackage{unicode-math}
  \defaultfontfeatures{Scale=MatchLowercase}
  \defaultfontfeatures[\rmfamily]{Ligatures=TeX,Scale=1}
\fi
\usepackage{lmodern}
\ifPDFTeX\else  
    % xetex/luatex font selection
\fi
% Use upquote if available, for straight quotes in verbatim environments
\IfFileExists{upquote.sty}{\usepackage{upquote}}{}
\IfFileExists{microtype.sty}{% use microtype if available
  \usepackage[]{microtype}
  \UseMicrotypeSet[protrusion]{basicmath} % disable protrusion for tt fonts
}{}
\makeatletter
\@ifundefined{KOMAClassName}{% if non-KOMA class
  \IfFileExists{parskip.sty}{%
    \usepackage{parskip}
  }{% else
    \setlength{\parindent}{0pt}
    \setlength{\parskip}{6pt plus 2pt minus 1pt}}
}{% if KOMA class
  \KOMAoptions{parskip=half}}
\makeatother
\usepackage{xcolor}
\usepackage[margin=1in]{geometry}
\setlength{\emergencystretch}{3em} % prevent overfull lines
\setcounter{secnumdepth}{-\maxdimen} % remove section numbering
% Make \paragraph and \subparagraph free-standing
\makeatletter
\ifx\paragraph\undefined\else
  \let\oldparagraph\paragraph
  \renewcommand{\paragraph}{
    \@ifstar
      \xxxParagraphStar
      \xxxParagraphNoStar
  }
  \newcommand{\xxxParagraphStar}[1]{\oldparagraph*{#1}\mbox{}}
  \newcommand{\xxxParagraphNoStar}[1]{\oldparagraph{#1}\mbox{}}
\fi
\ifx\subparagraph\undefined\else
  \let\oldsubparagraph\subparagraph
  \renewcommand{\subparagraph}{
    \@ifstar
      \xxxSubParagraphStar
      \xxxSubParagraphNoStar
  }
  \newcommand{\xxxSubParagraphStar}[1]{\oldsubparagraph*{#1}\mbox{}}
  \newcommand{\xxxSubParagraphNoStar}[1]{\oldsubparagraph{#1}\mbox{}}
\fi
\makeatother


\providecommand{\tightlist}{%
  \setlength{\itemsep}{0pt}\setlength{\parskip}{0pt}}\usepackage{longtable,booktabs,array}
\usepackage{calc} % for calculating minipage widths
% Correct order of tables after \paragraph or \subparagraph
\usepackage{etoolbox}
\makeatletter
\patchcmd\longtable{\par}{\if@noskipsec\mbox{}\fi\par}{}{}
\makeatother
% Allow footnotes in longtable head/foot
\IfFileExists{footnotehyper.sty}{\usepackage{footnotehyper}}{\usepackage{footnote}}
\makesavenoteenv{longtable}
\usepackage{graphicx}
\makeatletter
\def\maxwidth{\ifdim\Gin@nat@width>\linewidth\linewidth\else\Gin@nat@width\fi}
\def\maxheight{\ifdim\Gin@nat@height>\textheight\textheight\else\Gin@nat@height\fi}
\makeatother
% Scale images if necessary, so that they will not overflow the page
% margins by default, and it is still possible to overwrite the defaults
% using explicit options in \includegraphics[width, height, ...]{}
\setkeys{Gin}{width=\maxwidth,height=\maxheight,keepaspectratio}
% Set default figure placement to htbp
\makeatletter
\def\fps@figure{htbp}
\makeatother

\makeatletter
\@ifpackageloaded{caption}{}{\usepackage{caption}}
\AtBeginDocument{%
\ifdefined\contentsname
  \renewcommand*\contentsname{Table of contents}
\else
  \newcommand\contentsname{Table of contents}
\fi
\ifdefined\listfigurename
  \renewcommand*\listfigurename{List of Figures}
\else
  \newcommand\listfigurename{List of Figures}
\fi
\ifdefined\listtablename
  \renewcommand*\listtablename{List of Tables}
\else
  \newcommand\listtablename{List of Tables}
\fi
\ifdefined\figurename
  \renewcommand*\figurename{Figure}
\else
  \newcommand\figurename{Figure}
\fi
\ifdefined\tablename
  \renewcommand*\tablename{Table}
\else
  \newcommand\tablename{Table}
\fi
}
\@ifpackageloaded{float}{}{\usepackage{float}}
\floatstyle{ruled}
\@ifundefined{c@chapter}{\newfloat{codelisting}{h}{lop}}{\newfloat{codelisting}{h}{lop}[chapter]}
\floatname{codelisting}{Listing}
\newcommand*\listoflistings{\listof{codelisting}{List of Listings}}
\makeatother
\makeatletter
\makeatother
\makeatletter
\@ifpackageloaded{caption}{}{\usepackage{caption}}
\@ifpackageloaded{subcaption}{}{\usepackage{subcaption}}
\makeatother

\ifLuaTeX
  \usepackage{selnolig}  % disable illegal ligatures
\fi
\usepackage{bookmark}

\IfFileExists{xurl.sty}{\usepackage{xurl}}{} % add URL line breaks if available
\urlstyle{same} % disable monospaced font for URLs
\hypersetup{
  pdftitle={ADS 506 --- Week 5 Submission: Storytelling with Shiny},
  pdfauthor={Duy-Anh Dang},
  colorlinks=true,
  linkcolor={blue},
  filecolor={Maroon},
  citecolor={Blue},
  urlcolor={Blue},
  pdfcreator={LaTeX via pandoc}}


\title{ADS 506 --- Week 5 Submission: Storytelling with Shiny}
\author{Duy-Anh Dang}
\date{2025-11-24}

\begin{document}
\maketitle


\subsection{Part I --- Links and Description
(required)}\label{part-i-links-and-description-required}

\begin{itemize}
\tightlist
\item
  \textbf{Deployed Shiny App URL:} \url{https://...}
\item
  \textbf{Code URL:} (choose one)

  \begin{itemize}
  \tightlist
  \item
    GitHub repository:
    \url{https://github.com/MtnDoob/506_Week5_Dang_DuyAnh/tree/main}
  \item
    or posit.cloud project (ensure instructor access):
    \url{https://posit.cloud/...}
  \end{itemize}
\end{itemize}

App Description: \emph{This interactive Shiny application provides
comprehensive time series forecasting for Australian wine sales data
spanning from 1980 to 1994. The app enables users to analyze historical
sales patterns across six wine categories (Fortified, Red, Rose,
Sparkling, Sweet White, and Dry White) and generate forecasts using
three sophisticated statistical models. Users can customize training
periods, compare model performance, and visualize predictions with
confidence intervals through an intuitive interface.}

App Features:

\begin{itemize}
\tightlist
\item
  \emph{Interactive Data Exploration} - Select multiple wine varietals
  and customize date ranges to focus on specific time periods and wine
  types of interest
\item
  \emph{Flexible Train/Test Splits} - Define custom training end dates
  to evaluate model performance on held-out validation data
\item
  \emph{Multiple Forecasting Models} - Automatically fits and compares
  three models: TSLM (trend and seasonal regression), ETS (exponential
  smoothing), and ARIMA (autoregressive integrated moving average)
\item
  \emph{Dynamic Forecast Visualization} - Interactive Plotly charts
  display historical data, forecasts with prediction intervals, and
  clear train/validation boundaries
\item
  \emph{Model Performance Metrics} - Comprehensive accuracy tables
  showing RMSE, MAE, and MAPE for both training and validation periods
  across all models
\item
  \emph{Detailed Model Specifications} - View the automatically selected
  parameters for each model (ETS states, ARIMA orders, seasonal
  components)
\item
  \emph{Adjustable Forecast Horizon} - Generate forecasts for 1-24
  months ahead to support short and medium-term planning
\end{itemize}

\subsection{Part II --- Data Story (≤ 2 pages
total)}\label{part-ii-data-story-2-pages-total}

Red wine sales exhibit substantially higher growth and volatility
compared to Dry white wine over the 1980-1994 period, with Red wine
showing a 95\% increase in average sales levels and wider seasonal
fluctuations. This divergence suggests fundamentally different demand
patterns that require distinct forecasting approaches---ETS models
capture Red wine's multiplicative seasonality more effectively, while
TSLM performs comparably for the steadier Dry white variety.
Understanding these differences is critical for inventory planning and
production allocation decisions.

Red wine sales nearly doubled from 1980 to 1994 with widening seasonal
swings characteristic of multiplicative seasonality, while Dry white
wine remained stable with consistent additive seasonal patterns. These
contrasting dynamics require different forecasting approaches---ETS
models for Red wine's proportional seasonality versus TSLM for Dry
white's stable patterns---with critical implications for capacity
planning and inventory management in the Australian wine industry.

\includegraphics{aus.png} Reproduction: App URL = ; Varietals = {[}Dry
white, Red{]}; Date Range = 1980-01-01 to 1994-12-01; Training End =
1991-12-01

The overview reveals fundamental differences between these varietals.
Dry white wine maintained stable sales (2,500-5,000 units) with
consistent seasonal amplitude throughout the 14-year period, indicating
market maturity and additive seasonality likely tied to summer
consumption peaks. In contrast, Red wine climbed from approximately
1,500 to over 3,000 units---a 100\% increase---with seasonal swings that
widened proportionally. This pattern is characteristic of multiplicative
seasonality, where percentage fluctuations remain constant but absolute
swings grow as the trend level rises. These patterns drive strategic
decisions across four domains. First, model selection must be
varietal-specific: Red wine requires exponential smoothing (ETS) to
capture proportional seasonal effects, while Dry white's additive
pattern suits simpler linear trend methods (TSLM). Second, capacity
planning should prioritize Red wine production investment and vineyard
expansion, while Dry white has reached market saturation and should
focus on efficiency optimization. Third, inventory management for Red
wine needs dynamic safety stock that scales with demand---buffers
adequate in 1985 are insufficient by 1994---whereas Dry white can
maintain static policies. Finally, Red wine's sustained growth signals
shifting consumer preferences that warrant investigation to inform
strategy for other varietals. The training split at December 1991
positions us to rigorously test whether models can predict the
accelerating growth and volatility observed in the final three years.

\begin{center}\rule{0.5\linewidth}{0.5pt}\end{center}




\end{document}
